%-------------------------------------------------------------------------------

% This file is part of code_saturne, a general-purpose CFD tool.
%
% Copyright (C) 1998-2022 EDF S.A.
%
% This program is free software; you can redistribute it and/or modify it under
% the terms of the GNU General Public License as published by the Free Software
% Foundation; either version 2 of the License, or (at your option) any later
% version.
%
% This program is distributed in the hope that it will be useful, but WITHOUT
% ANY WARRANTY; without even the implied warranty of MERCHANTABILITY or FITNESS
% FOR A PARTICULAR PURPOSE.  See the GNU General Public License for more
% details.
%
% You should have received a copy of the GNU General Public License along with
% this program; if not, write to the Free Software Foundation, Inc., 51 Franklin
% Street, Fifth Floor, Boston, MA 02110-1301, USA.

%-------------------------------------------------------------------------------

\programme{gradrc}\label{ap:gradrc}

\vspace{1cm}
%-------------------------------------------------------------------------------
\section*{Fonction}
%-------------------------------------------------------------------------------

Le but de ce sous-programme est de calculer, au centre des cellules, le gradient
d'une fonction scalaire, \'egalement connue au centre des cellules.
Pour obtenir la valeur du gradient, une m\'ethode it\'erative de
reconstruction pour les maillages non orthogonaux est mise en
\oe uvre~: elle fait appel \`a un d\'eveloppement limit\'e d'ordre 1 en espace
sur la variable, obtenu \`a partir de la
valeur de la fonction et de son gradient au centre de la cellule. Cette
m\'ethode,
choisie comme option par d\'efaut, correspond \`a \var{imrgra}\,=\,0 et est utilis\'ee pour le calcul
des gradients de toutes les grandeurs.

%%%%%%%%%%%%%%%%%%%%%%%%%%%%%%%%%%
%%%%%%%%%%%%%%%%%%%%%%%%%%%%%%%%%%
\section*{Discr\'etisation}
%%%%%%%%%%%%%%%%%%%%%%%%%%%%%%%%%%
%%%%%%%%%%%%%%%%%%%%%%%%%%%%%%%%%%

La m\'ethode est d\'ecrite \'a la section~\ref{sec:spadis:iteratif_gradient}.

\minititre{Remarque}
Pour les conditions aux limites en pression, un traitement particulier est mis
en  \oe uvre, surtout utile dans les cas o\`u un gradient de pression (hydrostatique
ou autre) n\'ecessite une attention sp\'ecifique aux bords, o\`u une condition
\`a la limite de type Neumann homog\`ene est g\'en\'eralement inadapt\'ee. Soit
$P_{F_{\,b_{\,ik}}}$ la  valeur de la pression \`a la face associ\'ee, que
l'on veut calculer.

On note que ~:
\begin{equation}\notag
\vect{I'F}_{\,b_{\,ik}} \,.\,(\grad P)_I = \vect{I'F}_{\,b_{\,ik}}
\,.\,\vect{G}_{\,c,i} = \overline{I'F}_{\,b_{\,ik}} \,.\left. \displaystyle\frac{\delta P}{\delta
n}\right|_{F_{\,b_{\,ik}}}
\end{equation}
avec les conventions pr\'ec\'edentes.\\
\paragraph{\bf Sur maillage orthogonal }
On se place dans le cas d'un maillage orthogonal , {\it i.e.} pour
toute cellule $\Omega_I$, $I$ et son projet\'e $I'$ sont identiques.
Soit $M_{\,b_{\,ik}}$ le milieu du segment $IF_{\,b_{\,ik}}$.\\
On peut \'ecrire les \'egalit\'es suivantes~:
\begin{equation}\notag
\begin{array}{ll}
P_{F_{\,b_{\,ik}}} & = P_{M_{\,b_{\,ik}}} + \overline{M_{\,b_{\,ik}}F_{\,b_{\,ik}}}\,.\left. \displaystyle\frac{\delta P}{\delta
n}\right|_{M_{\,b_{\,ik}}} +
\overline{M_{\,b_{\,ik}}F_{\,b_{\,ik}}}^{\,2}\,.\left.\displaystyle\frac{1}{2}\frac{{\delta}^{2} P}{\delta
n^2}\right|_{M_{\,b_{\,ik}}} + \mathcal{O}(h^3)\\
P_I & = P_{M_{\,b_{\,ik}}} + \overline{M_{\,b_{\,ik}}I}\,.
\left. \displaystyle\frac{\delta P}{\delta n}\right|_{M_{\,b_{\,ik}}} +
\overline{M_{\,b_{\,ik}}I}^{\,2}\,.\left.\displaystyle\frac{1}{2}
\frac{{\delta}^{2} P}{\delta n^2}\right|_{M_{\,b_{\,ik}}} + \mathcal{O}(h^3)
\end{array}
\end{equation}
avec $\overline{M_{\,b_{\,ik}}I} = - \overline{M_{\,b_{\,ik}}F_{\,b_{\,ik}}}$.\\
On obtient donc~:
\begin{equation}\label{Base_Gradrc_eq_orthogonal}
P_{F_{\,b_{\,ik}}} - P_I = \overline{IF}_{\,b_{\,ik}}\,.\left. \displaystyle\frac{\delta P}{\delta
n}\right|_{M_{\,b_{\,ik}}} + \mathcal{O}(h^3)
\end{equation}
Gr\^ace \`a la formule des accroissements finis :
\begin{equation}\label{Base_Gradrc_eq_derivee_normale}
\left. \displaystyle\frac{\delta P}{\delta n}\right|_{M_{\,b_{\,ik}}} =
\displaystyle\frac{1}{2}\left[\left. \displaystyle\frac{\delta P}{\delta
n}\right|_{I} +  \left. \displaystyle\frac{\delta P}{\delta
n}\right|_{F_{\,b_{\,ik}}}\right] + \mathcal{O}(h^2)
\end{equation}\\
On s'int\'eresse aux cas suivants :\\\\
\hspace*{0.5cm}{ $\bullet${\underline { condition \`a la limite de type Dirichlet}}}\\
$P_{F_{\,b_{\,ik}}} = P_{IMPOSE}$, aucun traitement particulier\\\\
\hspace*{0.5cm}{ $\bullet ${\underline { condition \`a la limite de type Neumann
homog\`ene}}}\\
On veut imposer :
\begin{equation}
\left. \displaystyle\frac{\delta P}{\delta n}\right|_{F_{\,b_{\,ik}}} = 0 + \mathcal{O}(h)
\end{equation}
On a~:
\begin{equation}\notag
\left. \displaystyle\frac{\delta P}{\delta n}\right|_{I} =
\displaystyle\left. \displaystyle\frac{\delta P}{\delta
n}\right|_{F_{\,b_{\,ik}}} + \mathcal{O}(h)
\end{equation}
et comme :
\begin{equation}
P_{F_{\,b_{\,ik}}} = P_I + \overline{IF}_{\,b_{\,ik}}\,.\left. \displaystyle\frac{\delta P}{\delta
n}\right|_I + \mathcal{O}(h^2)
\end{equation}
on en d\'eduit :
\begin{equation}
P_{F_{\,b_{\,ik}}} = P_I + \overline{IF}_{\,b_{\,ik}}\,.\left. \displaystyle\frac{\delta P}{\delta
n}\right|_{F_{\,b_{\,ik}}} + \mathcal{O}(h^2)
\end{equation}
soit~:
\begin{equation}
P_{F_{\,b_{\,ik}}} = P_I +  \mathcal{O}(h^2)
\end{equation}
On obtient donc une approximation d'ordre 1.\\
\paragraph{\bf Sur maillage non orthogonal}
Dans ce cas, on peut seulement \'ecrire~:\\
\begin{equation}
P_{F_{\,b_{\,ik}}}  = P_{I'} +
\displaystyle\frac{1}{2}\,\vect{I'F}_{\,b_{\,ik}}\,.\,[\,(\grad P)_{I'} + (\grad
P)_{F_{\,b_{\,ik}}}\,] + \mathcal{O}(h^3)
\end{equation}
\hspace*{0.5cm}{ $\bullet ${\underline { condition \`a la limite de type Dirichlet}}\\
$P_{F_{\,b_{\,ik}}} = P_{IMPOSE}$, toujours aucun traitement particulier\\\\
\hspace*{0.5cm}{ $\bullet ${\underline { condition \`a la limite de type Neumann
homog\`ene}}}\\
On veut :
\begin{equation}
\left. \displaystyle\frac{\delta P}{\delta n}\right|_{F_{\,b_{\,ik}}} = 0 + \mathcal{O}(h)
\end{equation}
ce qui entra\^\i ne :
\begin{equation}\label{Base_Gradrc_eq_ortho}
\vect{I'F}_{\,b_{\,ik}} \,.\,(\grad P)_{F_{\,b_{\,ik}}} = \mathcal{O}(h^2)
\end{equation}
On peut \'ecrire :
\begin{equation}\notag
(\grad P)_{I'} = (\grad P)_{F_{\,b_{\,ik}}} +  \mathcal{O}(h)
\end{equation}
d'o\`u~:
\begin{equation}
P_{F_{\,b_{\,ik}}} = P_{I'}  + \mathcal{O}(h^2)
\end{equation}
On obtient donc une approximation d'ordre 1.\\

\hspace*{0.5cm}{\bf $\bullet $ Conclusion }\\
On peut r\'ecapituler toutes ces situations {\it via} la formule :
\begin{equation}\notag
P_{F_{\,b_{\,ik}}}\,=\,P_{I'}
\end{equation}

